\documentclass[12pt,a4paper]{article}
\usepackage[utf8]{inputenc}
\usepackage[T1]{fontenc}
\usepackage{geometry}
\usepackage{hyperref}
\usepackage{enumitem}
\usepackage{array}
\setlist{nosep}
\usepackage{xcolor}
\usepackage{tcolorbox}
\n
\title{Automated Bitcoin Trading – Feinkonzept}
\author{Juri S.}
\date{May 10, 2025}

\begin{document}
\maketitle
\tableofcontents
\newpage

\section{Research Objectives}
\begin{enumerate}
  \item Identify statistically significant signals that indicate potential short-term price moves.
  \item Convert the most promising signal(s) into an executable rule-based strategy.
  \item Evaluate performance via back-testing and a four-week live test (16~Jun – 14~Jul 2025).
  \item Document robustness, limitations and possible improvements.
\end{enumerate}

\section{Contents of Thesis}
\subsection{Market Mechanics}
Outlining what Bitcoin is and what the value of Bitcoin represents. I go into the basics of auction theory and why it is important for understanding
how markets work. I also discuss the efficient market hypothesis and its
relevance to algorithmic trading.

\subsection{From data analysis to trading strategy}
I start by visualizing the data using simple line charts and scatter plots to identify intuitive patterns that appear statistically relevant.
These patterns are then tested for statistical significance using basic methods such as conditional probability comparisons and hypothesis testing, implemented in Python.
I plan to use several indicators and combine them into a trading strategy. 
Different strategies with various indicator combinations will be evaluated using backtesting on historical data and walk-forward analysis.

%out of sample => Data not used for optimization
%in sample => Data used for optimization

\vspace{0.5em}
\noindent\textit{\footnotesize%
\textbf{Indicator} is a statistical measure for current market conditions.
}







\subsection{Risk \& Position Size}
Dive into risk management on how I handle position size and how I set the stop-loss and take-profit levels. 

\vspace{0.5em}
\noindent\textit{\footnotesize%
\textbf{stop-loss} Is an order to sell a position if the price drops below a certain level which marks the invalidation of your initial ideas of the trade.
\textbf{Take-profit} Is an order to sell a position if the price rises above a certain level which marks the end of the trade.
}




\newpage

\subsection{Automation and Test Run}
Dive into how I automate the trading process and live singal calculation. I explain how the bot
is set up to run automatically, how signals are processed, and how orders
are sent to the exchange.
Go into the basics of the mechanics behind the bot and what kind of libaries I used in python. Where it is running on a PaaS.
I also cover how I monitor the system, log trades
and errors, and set up alerts for any issues. Finally, I explain how I run a
test phase (paper trading) to make sure everything works as expected before
using real money.

\vspace{0.5em}
\noindent\textit{\footnotesize%
\textbf{PaaS} is a cloud platform as a service. It is a platform that allows you to run your own software in the cloud remotely.}


\section*{3\hspace{1em}Project Timeline}
\addcontentsline{toc}{section}{3\hspace{1em}Project Timeline}
\vspace{-0.5em}
\noindent\rule{\textwidth}{0.5pt}

\begin{tabular}{|>{\raggedright\arraybackslash}p{7cm}|>{\raggedright\arraybackslash}p{7cm}|}
\hline
\textbf{Phase} & \textbf{Weeks / Dates}\\ \hline
Exploratory analysis \& signal ideation & KW~17 -- KW~21 (24~Apr -- 26~May)\\ \hline
Parameter tuning \& dry-run (paper trading) & KW~22 -- KW~25 (27~May -- 23~Jun)\\ \hline
Live deployment (real funds) & KW~28 -- KW~31 (8~Jul -- 4~Aug)\\ \hline
Result analysis \& thesis writing & KW~32 (5~Aug -- 11~Aug)\\ \hline
Editing, final layout, submission & KW~33 -- KW~35 (12~Aug -- 1~Sep)\\ \hline
\end{tabular}

\vspace{0.5em}

\begin{tcolorbox}[colback=blue!10!white, colframe=blue!10!white, boxrule=0pt, arc=0pt, left=0pt, right=0pt, top=0pt, bottom=0pt]
\textit{This timeline could change and is just a rough estimate.}
\end{tcolorbox}

\section{Expected Delivery}
\begin{itemize}
  \item Python code for data handling, back-testing and live execution in a GitHub repository.
  \item Example datasets already available at
        \href{https://github.com/AJslashTracey/OBDeltaData}{GitHub repository}.
  \item A statistical report comparing other strategies I tested and other signals I considered.
  \item Final thesis detailing methodology, live results and critical reflection.
\end{itemize}

\end{document}
