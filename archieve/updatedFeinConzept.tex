\documentclass[12pt,a4paper]{article}
\usepackage[utf8]{inputenc}
\usepackage[T1]{fontenc}
\usepackage{geometry}
\usepackage{hyperref}
\usepackage{enumitem}
\usepackage{array}
\setlist{nosep}
\usepackage{xcolor}
\usepackage{tcolorbox}

\title{Automated Bitcoin Trading – Feinkonzept}
\author{Juri S.}
\date{May 10, 2025}

\begin{document}
\maketitle
\tableofcontents
\newpage

\section{Research Objectives}
\begin{enumerate}
  \item Identify statistically significant signals that signal potential short-term price moves.
  \item Convert the most promising signal(s) into an executable rule-based strategy.
  \item Evaluate performance via back-testing and a four-week live test (16~Jun – 14~Jul 2025).
  \item Document robustness, limitations and possible improvements.
\end{enumerate}

\section{Contents of Thesis}
\subsection{Market Mechanics}
Here I explain what Bitcoin is and what the value of Bitcoin represents. I then
cover the basics of auction theory and why it is important for understanding
how markets work. Finally, I discuss the efficient market hypothesis and its
relevance to algorithmic trading.

\subsection{Exploratory Data Analysis}
Here I describe the process of exploring the data. I look at the data in plain
terms, plot the bid- and ask-side deltas, check how often and how strongly
they move, and see whether large changes line up with later price moves. I
also use simple tools such as line charts and histograms to spot patterns that
seem to be statistically relevant.

\subsection{Risk \& Position Size}
Here I explain how I handle risk and position size. How I calculate the
position size and how I set the stop-loss and take-profit levels. How I came
up with the numbers and how I adjusted them.

\subsection{Automation and Test Run}
Here I describe how I automate the trading process. I explain how the bot
is set up to run automatically, how signals are processed, and how orders
are sent to the exchange. I also cover how I monitor the system, log trades
and errors, and set up alerts for any issues. Finally, I explain how I run a
test phase (paper trading) to make sure everything works as expected before
using real money.

\subsection{Termonology}

\section*{3\hspace{1em}Project Timeline}
\addcontentsline{toc}{section}{3\hspace{1em}Project Timeline}
\vspace{-0.5em}
\noindent\rule{\textwidth}{0.5pt}

\begin{tabular}{|>{\raggedright\arraybackslash}p{7cm}|>{\raggedright\arraybackslash}p{7cm}|}
\hline
\textbf{Phase} & \textbf{Weeks / Dates}\\ \hline
Exploratory analysis \& signal ideation & KW~17 -- KW~21 (24~Apr -- 26~May)\\ \hline
Parameter tuning \& dry-run (paper trading) & KW~22 -- KW~25 (27~May -- 23~Jun)\\ \hline
Live deployment (real funds) & KW~28 -- KW~31 (8~Jul -- 4~Aug)\\ \hline
Result analysis \& thesis writing & KW~32 (5~Aug -- 11~Aug)\\ \hline
Editing, final layout, submission & KW~33 -- KW~35 (12~Aug -- 1~Sep)\\ \hline
\end{tabular}

\vspace{0.5em}

\begin{tcolorbox}[colback=blue!10!white, colframe=blue!10!white, boxrule=0pt, arc=0pt, left=0pt, right=0pt, top=0pt, bottom=0pt]
\textit{This timeline could change and is just a rough estimate.}
\end{tcolorbox}

\section{Expected Delivery}
\begin{itemize}
  \item Python code for data handling, back-testing and live execution in a GitHub repository.
  \item Example datasets already available at
        \href{https://github.com/AJslashTracey/OBDeltaData}{GitHub repository}.
  \item A statistical report comparing other strategies I tested and other signals I considered.
  \item Final thesis detailing methodology, live results and critical reflection.
\end{itemize}

\end{document}
