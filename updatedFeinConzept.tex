\documentclass[a4paper,12pt]{article}
\usepackage{graphicx}
\usepackage{amsmath}
\usepackage{booktabs}
\usepackage{hyperref}
\usepackage{tocloft}
\setlength{\cftbeforesecskip}{6pt}

\title{Automated Bitcoin Trading – Feinkonzept}
\author{Juri S.}
\date{\today}

\begin{document}
\maketitle
\tableofcontents
\newpage

%────────────────────────────────────────────────────────
\section{Problem Statement}

The Bitcoin market is liquid but I think it still includes short‑term inefficiencies
caused by leverage, liquidations and order‑flow imbalances.  
The aim of this thesis is to \textbf{develop, validate and live‑test a
realistic algorithmic strategy} that can exploit such any kind of market inefficiencies
 within the available project timeline, if there even are any kind of inefficiencies.

%────────────────────────────────────────────────────────
\section{Research Objectives}

\begin{enumerate}
  \item Identify statistically significant signals that
        signal potential short‑term price moves.
  \item Convert the most promising signal(s) into an executable
        rule‑based strategy.
  \item Evaluate performance via back‑testing and a four‑week live test
        (16 Jun – 14 Jul 2025).
  \item Document robustness, limitations and possible improvements.
\end{enumerate}

%────────────────────────────────────────────────────────
\section{Data \& Tools}

Depth‑weighted order‑book snapshots at a \textbf{1‑minute resolution}
and corresponding trade prints are collected via the Hyperliquid API
(March\,--\,May\,2025).  All analysis is performed in Python using
\emph{pandas} for data handling, \emph{vectorbt} for parameter sweeps
and back‑testing, and \emph{Plotly} for interactive visualisation.
%────────────────────────────────────────────────────────
\section{Methodology}

\subsection{Exploratory Data Analysis}

The first step is to look at the data in plain terms: plot the bid‑ and
ask‑side deltas, check how often and how strongly they move, and see
whether large changes line up with later price moves.  Simple tools
such as line charts, histograms, and rolling correlations will show
which signals look promising.  The goal is not to prove a theory at
this stage, but to spot patterns that are \textbf{statistically
relevant} and therefore worth testing in a trading rule.



\subsection{Risk \& Position Size}

I will test a few stop‑loss and take‑profit settings and pick the safest
ones.  Each trade is kept small enough so that no more than a tiny
fraction of the account can be lost if things go wrong.  An automatic
check blocks any order that would risk too much.

\subsection{Automation and Test Run}

First, the whole bot will run in \emph{paper mode} (no real money) for a
few days to make sure signals, orders and logs work correctly.  If that
test passes, the bot will switch to a small real‑money account for a
four‑week trial.  Every trade, delay and profit or loss is saved
automatically, and a safety script can stop trading if preset limits are
hit.


%────────────────────────────────────────────────────────
\section{Project Timeline}

\begin{tabular}{ll}
\toprule
\textbf{Phase} & \textbf{Weeks / Dates} \\
\midrule
Exploratory analysis \& signal ideation      & KW 17 – KW 21 (24 Apr – 26 May) \\
Parameter tuning \& dry‑run (paper trading)  & KW 22 (27 May – 2 Jun) \\
Live deployment (real funds)                 & KW 25 – KW 28 (16 Jun – 14 Jul) \\
Result analysis \& thesis writing           & KW 29 – KW 33 (15 Jul – 18 Aug) \\
Editing, final layout, submission           & KW 34 – KW 36 (19 Aug – 5 Sep) \\
\bottomrule
\end{tabular}


This timeline could change and is just a rough estimate.

%────────────────────────────────────────────────────────
\section{Expected Deliverables}

\begin{itemize}
  \item Clean, annotated Python code for data handling, back‑testing and
        live execution.
  \item A statistical report comparing candidate signals.
  \item Final thesis  detailing methodology, live results and
        critical reflection.
\end{itemize}

%────────────────────────────────────────────────────────


\end{document}
