\documentclass[a4paper,12pt]{article}
\usepackage{graphicx}
\usepackage{amsmath}
\usepackage{booktabs}
\usepackage{hyperref}
\usepackage{tocloft}
\setlength{\cftbeforesecskip}{6pt}

\title{Automated Bitcoin Trading – Feinkonzept}
\author{Juri S.}
\date{\today}

\begin{document}
\maketitle
\tableofcontents
\newpage

%────────────────────────────────────────────────────────
\section{Problem Statement}

The Bitcoin market is liquid but still exhibits short‑term inefficiencies
caused by leverage, liquidations and order‑flow imbalances.  
The aim of this thesis is to \textbf{develop, validate and live‑test a
realistic algorithmic strategy} that can exploit such any kind of market inefficiencies
 within the available project timeline.

%────────────────────────────────────────────────────────
\section{Research Objectives}

\begin{enumerate}
  \item Identify statistically significant signals that
        signal potential short‑term price moves.
  \item Convert the most promising signal(s) into an executable
        rule‑based strategy.
  \item Evaluate performance via back‑testing and a four‑week live test
        (16 Jun – 14 Jul 2025).
  \item Document robustness, limitations and possible improvements.
\end{enumerate}

%────────────────────────────────────────────────────────
\section{Data \& Tools}

Depth‑weighted order‑book snapshots at a \textbf{1‑minute resolution}
and corresponding trade prints are collected via the Hyperliquid API
(March\,--\,May\,2025).  All analysis is performed in Python using
\emph{pandas} for data handling, \emph{vectorbt} for parameter sweeps
and back‑testing, and \emph{Plotly} for interactive visualisation.
%────────────────────────────────────────────────────────
\section{Methodology}

\subsection{Exploratory Data Analysis}

Initial work will focus on descriptive statistics of depth‑weighted
bid/ask deltas, volatility clustering and liquidation events.  
Rolling correlations and lead–lag plots help decide which variables are
worth modelling.

\subsection{Signal Development}

Several candidate signals will be tested, e.g.

\begin{itemize}
  \item Order‑flow‑imbalance (OFI) quantiles
  \item Depth‑specific delta momentum
  \item Combined price–volume features
\end{itemize}

Each candidate is evaluated with walk‑forward back‑tests to avoid
look‑ahead bias.

\subsection{Risk \& Position Sizing}

Stop‑loss / take‑profit distances will be tuned via grid‑search.
Maximum capital at risk per trade is capped at 1 \% of account equity.

\subsection{Live Deployment}

The best performing rule set will be deployed with real capital
(\$500) for four weeks. All trades, latency metrics and
PnL are logged automatically.

%────────────────────────────────────────────────────────
\section{Project Timeline}

\begin{tabular}{ll}
\toprule
\textbf{Phase} & \textbf{Weeks / Dates} \\
\midrule
Exploratory analysis \& signal ideation      & KW 17 – KW 21 (24 Apr – 26 May) \\
Parameter tuning \& dry‑run (paper trading)  & KW 22 (27 May – 2 Jun) \\
Live deployment (real funds)                 & KW 25 – KW 28 (16 Jun – 14 Jul) \\
Result analysis \& thesis writing           & KW 29 – KW 33 (15 Jul – 18 Aug) \\
Editing, final layout, submission           & KW 34 – KW 36 (19 Aug – 5 Sep) \\
\bottomrule
\end{tabular}

%────────────────────────────────────────────────────────
\section{Expected Deliverables}

\begin{itemize}
  \item Clean, annotated Python code for data handling, back‑testing and
        live execution.
  \item A statistical report comparing candidate signals.
  \item Final thesis (~30 pages) detailing methodology, live results and
        critical reflection.
\end{itemize}

%────────────────────────────────────────────────────────
\section{Risk Management}

\begin{itemize}
  \item \textbf{Market risk:} fixed stop‑loss and kill‑switch at
        –3 \% account equity.
  \item \textbf{Over‑fitting:} walk‑forward validation and
        out‑of‑sample hold‑out.
  \item \textbf{Operational risk:} cloud server with automatic restart
        and daily data backup.
\end{itemize}

%────────────────────────────────────────────────────────


%────────────────────────────────────────────────────────


\end{document}
