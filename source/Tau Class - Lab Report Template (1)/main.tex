%%%%%%%%%%%%%%%%%%%%%%%%%%%%%%%%%%%%%%%%%%%%%%%%%%%%%%%%%%
% Tau LaTeX Class
% Trading Bot Thesis Example (with Orderbook image)
%%%%%%%%%%%%%%%%%%%%%%%%%%%%%%%%%%%%%%%%%%%%%%%%%%%%%%%%%%
\documentclass[9pt,a4paper,twocolumn,twoside]{tau-class/tau}
\usepackage[english]{babel}
\usepackage{float}  % for [H] if needed
\usepackage{subcaption} % only if you're using sub-figures

%----------------------------------------------------------
% TITLE
%----------------------------------------------------------
\title{Automated Trading Bot: Feinkonzept}

%----------------------------------------------------------
% AUTHORS, AFFILIATIONS AND PROFESSOR
%----------------------------------------------------------
\author[a,1]{Juri Stoffers}
\affil[a]{Gymthun}
\professor{Begleitperson: Dr. Ostrin Geoffrey}

%----------------------------------------------------------
% FOOTER INFORMATION
%----------------------------------------------------------
\institution{Institution Name}
\footinfo{Matura Project}
\theday{April 2025}
\leadauthor{S.}
\course{Creative Commons CC BY 4.0}

%----------------------------------------------------------
% If you remove or comment out this entire block, the abstract is gone:
% \begin{abstract}
% Here was the abstract text
% \end{abstract}

% If you remove or comment out \keywords, that also disappears:
% \keywords{Trading Bot, Backtesting, System Design, Automated Trading}

%----------------------------------------------------------
\begin{document}

\maketitle
\thispagestyle{firststyle}

% \tableofcontents
% \linenumbers

%----------------------------------------------------------
\section{Define My Goals \& Requirements}

\taustart{} I start by deciding what kind of trading I want to pursue—momentum, 
arbitrage, or another strategy. I set clear targets for profits and acceptable 
losses, and choose the markets I'll focus on (crypto, stocks, or forex). 
I also consider performance requirements like speed, reliability, and 
implementation complexity.

%----------------------------------------------------------
\section{Outline My System Design}

\subsection{Data Sources}
I gather both real-time and historical market data to feed my bot, ensuring 
I have enough coverage for backtesting and live operations.

\begin{figure}[H]
    \centering
    \includegraphics[width=0.85\columnwidth]{figures/newplot.png}
    \caption{Current Data set visualization}
    \label{fig:Dataset}
\end{figure}

\subsection{Signal Processing}
I develop modules to analyze data for trade signals, whether it's technical 
indicators (e.g., EMA, MACD) or more advanced methods (machine learning). 


\subsection{Efficient Market Hypothesis}
Is it possible to abstract money from the market, because if I would find an edge,
wouldn't it already be exploited so priced in?


\subsection{Order Management}
I connect to broker APIs so my bot can execute trades automatically. This 
includes handling order placement, cancellation, and partial fills.

\subsection{Risk Management}
Safety features like stop-loss orders and position sizing help me control 
potential drawdowns. I might use trailing stops or percentage-based 
position sizing.

\subsection{Monitoring}
I set up real-time tracking and logging to catch issues before they escalate, 
and to log trades for auditing performance over time.

%----------------------------------------------------------
\section{Backtesting \& Simulation}

Before going live, I test my strategy on historical data to refine it and 
uncover potential flaws. I also run a simulated (paper-trading) environment 
to confirm the bot’s logic without risking real funds. 

\subsection{Performance Thresholds}
For example, a Sharpe ratio below 1.5 might be more realistic, whereas 
anything above that could indicate overfitting or unrealistic assumptions.

%----------------------------------------------------------
\section{Implementation \& Operation}

I choose a programming language (e.g., Python) for its extensive libraries 
and community support. I build the system in modular parts so I can update 
individual sections without overhauling the entire setup. Finally, I 
integrate security measures like API encryption and strict access 
controls, while staying aware of regulatory requirements.

\noindent
This plan covers key areas: strategy, system design, testing, and secure 
live operation, giving me a roadmap for a trading bot that could potentially 
be profitable in real markets.

%----------------------------------------------------------
\section{Feinkonzept Notes (Additional Topics)}

\subsection{Data Creation}
\begin{itemize}
  \item Data sources
  \item Creating a good timeseries dataset
  \item Hosting data creation process
  \item External datasources for out-of-sample testing
\end{itemize}

\subsection{What is Bitcoin}
A form of digital currency that uses blockchain technology to support transactions between users on a decentralized network.

\subsection{How is the Price of Bitcoin Determined (Auction Principle)}

Below is an example “orderbook” image illustrating how bids and asks are 
arranged in the marketplace:

\begin{figure}[H]
    \centering
    \includegraphics[width=0.85\columnwidth]{figures/orderbook.png}
    \caption{Example orderbook visualization, showing bid/ask levels.}
    \label{fig:orderbook}
\end{figure}

\subsection{Backtesting Details}
\begin{itemize}
  \item Testing different kinds of strategies (Sample vs Out-of-sample)
  \item Sharpe ratio threshold (e.g., below 1.5 = more realistic)
  \item Real-time paper trading test
  \item Outline which indicators I’ll be using
\end{itemize}

\subsection{Outlining My System Design (Extra Points)}
\begin{itemize}
  \item Programming language of choice (why?)
  \item Setting up system in modular parts
  \item Safety measures
  \item Reliability of external APIs
  \item Broker I'm planning to use
  \item Latency considerations
  \item Order management details
  \item Define running period for Matura levy, and timeline up to final presentation
\end{itemize}

\section{Conclusion}
This Feinkonzept provides a high-level plan for developing and deploying 
an automated trading system. Each section—goals, design, backtesting, 
and implementation—helps ensure both theoretical soundness and practical 
feasibility.

%----------------------------------------------------------
\end{document}
